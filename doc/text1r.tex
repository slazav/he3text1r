\documentclass[a4paper]{article}

\usepackage{amssymb}
\usepackage{euscript}
\usepackage{graphicx}
\graphicspath{{pics/}}
\usepackage{color}
\usepackage{epsfig}
\usepackage{fullpage}
\usepackage{verbatim}

\begin{document}

\title{{\tt text1r} --- calculate 1D radial texture}
\date{\today}
\author{}
\maketitle

\def\sp{\sqrt{5}}
\def\st{\sqrt{3}}
\def\ddd#1#2{\frac{\partial #1}{\partial #2}}

This functions calculate 1D texture in $^3$He-B in a cylindrically
symmetric geometry. Code and ideas came from ROTA programs (by J.Kopu,
S.Autti, ...). Interfaces for C, F, F90, Matlab, and Octave languages
are available.

Sum of following energy terms is minimized:

\def\bn{{\bf n}}
\def\bH{{\bf H}}
\def\divn{(\nabla\cdot\bn)}
\def\rotn{(\nabla\times\bn)}
\def\grn{\nabla\bn}
Magnetic energy an a uniform field along $z$ axis:
$$
F_{DH} = - a \int_V (\bn\cdot\bH)^2 = -a H_z^2 \int_V \sin^2\beta_N
$$

Gradient energy:
$$
F = \frac{1}{13}\ \frac{\xi_H^2}{R^2}\int_V  [ 4 (4+\delta) (\grn)^2
  - (2+\delta) / 2  (\st \divn + \sp n \rotn^2 ]
$$

Spin-orbit energy for precessing magnetization:
$$
F = \chi \frac{\Omega^2_B}{\gamma^2} \int_V  \Psi^2 \sin^2\beta_N
$$

Superflow energy:
$$
F_{HV} = - \lambda_{HV} \int_V [\bH \cdot R \cdot ({\bf v_s-v_n})]^2
$$

Vortex energy:
$$
F = \frac15\ \frac{\lambda}{\Omega} \int_V  \Omega_v [\bH \cdot R \cdot {\bf l_v}]^2
$$

Surface energy:
$$
F = - \frac{5}{16}\ \frac{d}{aR} \int_S (\sp n_z n_r - \st n_f)^2
$$
$$
F = \frac{2}{13}\ \frac{\xi_H^2}{R^2} \int_S ( 2 (2 + \delta) - \lambda_{SG})  \sin^2\beta
$$

Angle $\theta$ in order parameter is fixed at $\cos^{-1}(-1/4)$.


\section*{Usage}

\subsection*{fortran}

A global structure is used to keep parameters and results. Number of points is
hardcoded into the program.


To specify textural parameters a global structure {\tt\bf text1r\_pars}
is used:

\begin{verbatim}

rr(MAXN)  // r grid [cm]
an(MAXN)  // azimuthal angle of n vector [degrees]
bn(MAXN)  // polar angle of n vector [degrees]
a         // Textural dipole-field parameter a [erg/cm^3 G^-2]
H         // Magnetic field [G]
R   // Cell radius [cm]
lg1 // Textural parameter lambda_g1 [erg/cm]
lg2 // Textural parameter lambda_g2 [erg/cm]

lhv // Textural parameter lambda_HV [erg/cm3 1/G2 1/(cm/s)^2]
lsg // Textural parameter lambda_SG [???]

ld  // Textural parameter lambda_D [erg/cm^3]
lo  // Textural parameter lambda/omega [s/rad]

d   // Textural surface parameter d [erg/cm^2 G^-2]

vr(MAXN)  // velocity profile
vz(MAXN)
vf(MAXN)

lr(MAXN)  // vortex polarization
lz(MAXN)
lf(MAXN)

w(MAXN)      // vortex dencity, ??
apsi(MAXN)   // magnon wavefunction amplitude

\begin{verbatim}
struct text1r_pars_t{
  double alpha_n[MAXN], beta_n[MAXN]; /* azimuthal and polar angles of n vector
                                         (initial conditions and final result) */
} text1r_pars;
\end{verbatim}

In fortran one sholud use header file {\tt text1r.fh} where common block

Function {\tt\bf text1r\_pars\_init} is used to initialize {\tt\bf text1r\_pars}
structure.

\eject
\section*{Technical details}

\subsection*{Program structure}

\begin{itemize}
\item {\tt esurf} and {\tt ebulk} subroutines calculate surface and
bulk energy and its derivatives as a function of texture ($\alpha$ and
$\beta$) and texture gradients ($\partial\alpha/\partial r$,
$\partial\beta/\partial r$ etc.). Texture can be represented in various
forms ($\alpha, \beta$, or $\bf n$, or $R_{ij}$).

\item {\tt egrad} subroutine calculates total energy as integral of bulk energy over volume
plus integral of surface energy over surface and its derivatives as a
function of texture and texture gradients at the whole grid.

\item {\tt mfunc} is a wrapper for {\tt egrad}. Texture is represented
as 1-d array sutable for minimization (see below).

\item {\tt x2text} and {text2x} subroutines convert two representations of the texture.

\item {\tt minimize} subroutine does minimization.
\end{itemize}


\subsection*{Texture representation for minimization}

We don't want to minimize directly $F(n_r, n_z, n_f)$ because additional
condition $n_i n_i = 1$ should be taken into account. We also don't want
to minimize $F(\alpha, \beta$) because if $\beta=0$ then $\alpha$
is not defined.

One posibility is to use a projection of the $\bf n$ sphere into a plane
$z=0$ from a $z=-1, r=0$ point:

$$
u=\frac{n_r}{1+n_z}, \qquad  v=\frac{n_f}{1+n_z}.
$$

Inverse transformation is
$$
n_z = \frac{1-u^2-v^2}{1+u^2+v^2},\qquad
n_r = \frac{2u}{1+u^2+v^2},\qquad
n_f = \frac{2v}{1+u^2+v^2},\qquad
$$

We also need
$$
\ddd{E}{u} = \ddd{E}{\alpha}\ddd{a}{u} + \ddd{E}{\beta}\ddd{b}{u}
$$
$$
\ddd{E}{v} = \ddd{E}{\alpha}\ddd{a}{v} + \ddd{E}{\beta}\ddd{b}{v}
$$

We don't want to minimaze $r=0$ point where $n_z=1$ because of cyllindrical symmetry.
\end{document}

